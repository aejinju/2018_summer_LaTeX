\documentclass{article}

\usepackage{kotex}
\usepackage{graphicx}
\usepackage{subcaption}

\begin{document}

	Figure~\ref{fig:coffee}은 계수의 합이 1 이하인 예이다.
	
	% 여러개의 그림 넣기
	% \linewidth의 계수의 합은 1 이하여야 함
	\begin{figure}[h!]
  		\centering
  		\begin{subfigure}[b]{0.4\linewidth}
    		\includegraphics[width=\linewidth]{coffee.jpg}
    		\caption{Coffee.}
  		\end{subfigure}
  		\begin{subfigure}[b]{0.4\linewidth}
    		\includegraphics[width=\linewidth]{coffee.jpg}
    		\caption{More coffee.}
  		\end{subfigure}
  		\caption{The same cup of coffee. Two times.}
  		\label{fig:coffee}
	\end{figure}

	Figure~\ref{fig:coffee3}은 계수의 합이 1 이상인 예이다.
	
	% \linewidth의 계수의 합이 1 이상인 경우
	\begin{figure}[h!]
  		\centering
  		\begin{subfigure}[b]{0.2\linewidth}
    		\includegraphics[width=\linewidth]{coffee.jpg}
     		\caption{Coffee.}
  		\end{subfigure}
  		\begin{subfigure}[b]{0.2\linewidth}
    		\includegraphics[width=\linewidth]{coffee.jpg}
    		\caption{More coffee.}
  		\end{subfigure}
  		\begin{subfigure}[b]{0.2\linewidth}
    		\includegraphics[width=\linewidth]{coffee.jpg}
    		\caption{Tasty coffee.}
  		\end{subfigure}
  		\begin{subfigure}[b]{0.5\linewidth}
    		\includegraphics[width=\linewidth]{coffee.jpg}
    		\caption{Too much coffee.}
  		\end{subfigure}
  		\caption{The same cup of coffee. Multiple times.}
  		\label{fig:coffee3}
	\end{figure}
	
\end{document}