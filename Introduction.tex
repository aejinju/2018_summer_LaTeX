\chapter{서론}	
		
	토마 피케티(Thomas Piketty)\cite{piketty:1}\cite{piketty:2}가 
	저서 "21세기 자본"에서 소득과 부의 
	불평등 심화 문제를 제기한 것을 계기로 경제적 불평등 문제는 
	최근 세계적인 관심을 받고 있다. 
	경제정의실천시민연합이 경제전문가들을 대상으로 실시한 설문조사에서
	'현재 우리 경제가 직면한 가장 큰 문제점'으로 
	응답자의 46.7\%가 '경제양극화 및 소득불평등'을 지적하는 등
	경제적 불평등 문제는 국내에서도 주목을 받고 있다. 
	
	이러한 상황에서 공공기관이 작성하는 소득불평등지수에 관심을 갖는 것은 
	자연스러운 일이다.
	한국의 경우에는 정부기관인 통계청이 매년 지니계수, 
	5분위배율\footnote{소득 상위20\%(5분위)계층의 평균소득을 
	소득 하위20\%(1분위)계층의 평균소득으로 나눈 값임.}, 
	상대적 빈곤율\footnote{중위소득 50\% 미만 계층의 비율임.} 
	등의 소득불평등지수를 작성 $\cdot$ 발표하고 있다. 
	최근 6년간의 발표를 보면 가처분소득
	\footnote{가처분소득 = 시장소득 + 공적 이전소득 - 공적 비소비지출 \\
			  시장소득 = 근로소득 + 사업소득 + 재산소득 + 사적 이전소득 \\
			  공적 이전소득 : 국민연금, 생계급여, 실업급여, 장애수당, 산재급여 등 \\
			  공적 비소비지출 : 조세, 사회보험료 등.}
	기준 지니계수는 2008년에 0.314에서 꾸준히 감소해 2013년에는 0.302에 이르렀고,
	5분위배율과 상대적 빈곤율 역시 감소하는 추세를 보여 소득불평등이 개선되고 
	있는 것으로 나타났다.
	
	...
